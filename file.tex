%% ****** Start of file template.aps ****** %
%%
%%
%%   This file is part of the APS files in the REVTeX 4 distribution.
%%   Version 4.0 of REVTeX, August 2001
%%
%%
%%   Copyright (c) 2001 The American Physical Society.
%%
%%   See the REVTeX 4 README file for restrictions and more information.
%%
%
% For Phys. Rev. appearance, change preprint to twocolumn.
% Choose pra, prb, prc, prd, pre, prl, prstab, or rmp for journal
%  Add 'draft' option to mark overfull boxes with black boxes
%
%
% To compile: latex prl.tex ; dvips prl.dvi -o prl.ps ; ps2pdf prl.ps
%
\documentclass[aps,prl,twocolumn,superscriptaddress,groupedaddress]{revtex4}
\usepackage{graphicx}  % needed for figures
\usepackage{dcolumn}   % needed for some tables
\usepackage{bm}        % for math
\usepackage{amssymb}   % for math
\bibdata{spin-qubits}

\begin{document}


\title{Semiconductor Spin Qubits}
\author{E.~Aaron}

\date{\today}


\begin{abstract}
Despite being inherently quantum, the binary nature of spin suggests a convenient analogy with the classical idea of a bit, which holds information in the form of exactly one of two possible discrete values. This quality of spin has prompted a line of research attempting to make use of it as a quantum bit (qubit) in the context of the growing field of quantum computing. In the following, I summarize the state of this research as presented in sections I-IV of a recent review by Burkard et al.\cite{burkard_semiconductor_2021}
\end{abstract}

\maketitle

\section{Introduction}
The idea of a quantum computer has its origins as early as the 1980s\cite{feynman_simulating_1982} and has since exploded into an interdisciplinary field of study containing numerous subfields. This type of device operates on an entirely different paradigm from that of classical (deterministic) computers: the computer itself should be quantum, meaning that its information components (the qubits) can exist in superposed and entangled states up to the moment of measurement. Qubits can, in principle, be any system capable of existing and being measured in quantum states.

At its most basic, the operation of a quantum computer consists of three stages: qubits must be initialized to a stable and replicable initial state, they must be controllable via a standardized set of quantum operators or gates, and the results of measurements performed by such operators must be read out into a usable form, most likely through some conversion of the quantum state into a correlated electronic signal. Notably, qubits' quantum states must remain intact (coherent) throughout this process, despite being constantly surrounded by an interacting environment. In fact, qubits always decohere, and the challenge is simply to extend the so-called decoherence time long enough for the purposes of the calculation.

Much research has been poured into designing qubits to match these challenges. One subset of such designs are semiconductor-hosted qubits which hold information in their spin states. While these are not immune to the issues discussed, they have a few advantages over other candidate qubit types. First, the choice of spin as the information-carrying degree of freedom allows for a wide variety of quantum systems to be used, and provides flexibility in the spatial configurations allowed for the qubit. Second, spin qubits consist of only a few electrons or atoms, making them smaller than other types.

\section{Principles of operation}

\begin{figure}
\label{bands}
\includegraphics[scale=0.7]{bands.png}
\caption{\label{bands}Band structure schematic of a quantum well constructed from GaAs and AlGaAs\cite{bands}.}
\end{figure}

The positions of spins must be well confined in order for one to distinguish and control individual qubits. The unique band structure properties of semiconductor materials make them ideal for engineering such confinement. Fig. \ref{bands} illustrates how a semiconductor with a relatively small energy band gap, sandwiched between two layers of a larger band gap semiconductor, create a 1-dimensional trap (quantum well) for charge-carrying electrons and holes. In quantum dot (QD) based qubits, this approach is used to confine electrons to a 2D plane. To localize the electrons in the other 2 dimensions, a QD is constructed by placing gate electrodes on top of the plane, creating potential “walls” of variable height around a small (~100nm scale) region. The number of electrons occupying the dot, as well as the coupling between neighboring dots, can be controlled by the gate voltages. Quantization of current \cite{marder_condensed_2010} allows occupation to be set with single-electron precision.

\begin{figure}
\label{axes}
\includegraphics[scale=1.2]{axes.png}
\caption{\label{axes}Control axes of a Loss-DiVincenzo (LD) qubit\cite{burkard_semiconductor_2021}.}
\end{figure}

Control of a qubit’s spin state requires access to at least two control axes, as illustrated in Fig. \ref{axes}. These are any two independent degrees of freedom which influence spin. Typically, spin qubits employ a single-spin control axis and a multi-spin axis created by spin-spin coupling. Coupling (either within a single qubit or between two qubits) is provided by the exchange interaction, which arises as a consequence of exchange antisymmetry in fermionic wavefunctions. This interaction can be represented by the Hubbard model Heisenberg hamiltonian:
\begin{equation}
H_{Heis} = -J \vec S_1 \dot{} \vec S_2
\end{equation}
where J is the coupling constant between spins 1 and 2.

In addition to confinement of and coupling between different spins, quantum computing operations generally involve manipulations of individual spins. There are several avenues by which this can be realized physically. The first is the Zeeman interaction, which is a coupling of single spins to applied magnetic fields. Via this coupling, a uniformly applied magnetic field will break the energy degeneracy between single particle spin-up and spin-down states. In practice, this means that the states become energetically distinguishable, permitting their use as information-carrying 0 and 1 states. If an applied field is sufficiently localized — which is challenging in practice, but possible — this effect can also be used to modulate the energy state spaces of individual qubits.

The spin-orbit interaction (SOC),
\begin{equation}
\hat{H}_{SO}\propto \vec{S}\cdot \vec{L} \cite{zettili_quantum_2009},
\end{equation}
also provides a path to single-qubit control, though it is more subtle than that of the Zeeman effect. The coupling of spin to orbital angular momentum weakly ties the spin of an affected particle to its orbital motion. Thus, a large enough change in momentum can affect a particle’s spin state. In practice, this is implemented by applying large electric or magnetic field gradients which accelerate (or decelerate) the particle in its orbital path.
The hyperfine interaction (HF) can also become important. This coupling between electronic and nuclear spins,
\begin{equation}
\hat{H}_{HF}\propto \sigma_{elec} \cdot \sigma_{nuc}
\end{equation}
is particularly sensitive to the distance between the interacting spins, and thus becomes most important in qubits built from atoms, where electrons are bound to the nucleus.

While potentially useful for qubit control, any spin-sensitive interaction can also be a source of decoherence for spin qubits. This is especially true of exchange and hyperfine interactions, which are present in all materials due to the many-body nature of the lattice and are thus difficult to minimize. Spin qubit design requires a delicate balancing of these effects in order to produce optimal performance and decoherence times.

\section{Implementations}
Semiconductor spin qubits exist in several flavors. These are distinguished by the nature of the host system that confines the spins, and the number of spins needed to construct a single qubit. The most common hosts are quantum dots, but qubits can also be constructed in which spins are confined only by the host material lattice. The number of spins per qubit varies from one to, in principle, any number, though fewer spins minimizes the size and complexity of the system. Roughly in order of increasing complexity, the four main design categories of spin-based qubits are Loss-DiVincenzo, donor, singlet-triplet (ST), and exchange-only (EO).

The Loss-DiVincenzo qubit is named for its earliest proponents and consists of a single electron in a quantum dot. The critical interactions for this state are exchange and Zeeman, represented by the Hamiltonian
\begin{equation}
H=\frac{1}{2}\sum_i g_i \mu_b \vec B_i \cdot \vec\sigma_i + \frac{1}{4}\sum_{<i,j>}J_{ij}(t)\vec\sigma_i \cdot \vec\sigma_j,
\end{equation}
where the second sum is performed over nearest neighbor qubits. Its 0 and 1 states correspond to the (collapsed) spin-up and spin-down states of the electron. Prior to collapse, a single qubit’s spin state can be rotated via AC magnetic pulses, and multi-qubit operations can be performed by modulating the exchange coupling between neighboring dots. To read out the state of a collapsed qubit, magnetic fields are used to move the spin-up state to a conduction-level energy, allowing only qubits in that state to tunnel into the conduction band and be detected as current. This type of qubit is susceptible to decoherence from unwanted hyperfine coupling, as well as from any "background" spin present in the semiconductor host. It also requires fine spatial control of the magnetic fields used to create Zeeman coupling to single qubits.

The donor spin qubit is an exception to the quantum dot majority, consisting of a donor atom confined by a semiconductor host lattice. The Hamiltonian for two adjacent qubits (ignoring external fields) looks like
\begin{equation}
H_{12}=A_1\sigma^{1nuc}\cdot\sigma^{2elec} + A_2\sigma^{2nuc}\cdot\sigma^{2elec} + J\sigma^{1elec}\cdot\sigma^{2elec},
\end{equation}
where the A's are hyperfine coupling constants for atoms 1 and 2. The states making up the qubit state subspace are any two of the joint nuclear and electronic spin states, though the purpose of the donor electrons is primarily to provide easy access to the nuclear spins, allowing for indirect manipulation and readout. Strategically placed gate electrodes provide this control via hyperfine coupling, by moving electrons closer to or further from their nuclei. Readout is done by measuring the coupled electrons’ states via gates. Because the nucleus is tightly confined by the lattice, it is not as susceptible to spin-orbit decoherence as other types of qubits. However, background nuclear spins can interfere with a donor qubit’s state. This is minimized by using a semiconductor host with average nuclear spin as near to zero as possible (often highly purified Si). The spatial placement of donor atoms in the host lattice is difficult to do with high precision, which creates a challenge since the multi-qubit exchange interaction, depends heavily on interatomic separation. This type of qubit was still in early stages of research at the time of the paper’s writing, and a scalable array of working qubits had not yet been demonstrated.

\begin{figure}
\label{eofig}
\includegraphics[scale=0.25]{eo.png}
\caption{\label{eofig}The state space and control axes of an EO qubit\cite{burkard_semiconductor_2021}.}
\end{figure}

The singlet-triplet qubit, as its name suggests, occupies the singlet and triplet spin subspace of a two-electron system. The electrons are housed in two exchange-coupled quantum dots. The two states are nondegenerate in energy due to exchange, and thus form an acceptable 0-1 pair. In this design, entanglement makes manipulation of individual spins nontrivial. One solution is to employ the spin-orbit effect under a magnetic field gradient across the two dots, so that each is affected differently by SOC.  The resulting Hamiltonian is
\begin{equation}
\hat{H}_{ST}=J_{12}\frac{\sigma_z}{2}+\mu_b \Delta \left(g^{*}B^{z}\right)\frac{\sigma_x}{2}.
\end{equation}
To read out a collapsed state, one simply allows the Pauli principle to determine whether the two electrons occupy separate dots or the same dot. The locations of the electrons can then be detected by a charge sensor and correlated with their state. Variations of this type of qubit use a magnetically polarized triplet as the “1” state. Since the total spin quantum number of a two-electron entangled state is m=0, uniform magnetic fields cannot cause ST qubits to decohere, though any fluctuations in the applied gradient can still do so. Hyperfine coupling to semiconductor nuclei can also cause decoherence.

While smaller spin qubits require at least one single-spin degree of freedom to control the state, one can construct a qubit which operates exclusively on exchange operations. The cost of this simpler operation is a larger, more complex qubit. Exchange-only qubits contain at least three electrons in a triple QD (Fig. 3). The control axes are the exchange couplings $J_{12} $ and $J_{23}$. The three-body Hilbert space contains a subset of states which can be set purely by the two couplings, and these states are used to define the qubit. Multi-qubit operations also employ exchange, coupling spins across different qubits. Exchange couplings are needed for all operations, and thus are always nonzero somewhere in the system. This makes the probability of unwanted electrons tunneling into an EO qubit  also nonzero, creating a potential source of decoherence that is less problematic in other qubit types.

\section{Conclusion}
The four semiconductor spin qubits discussed are not an exhaustive list, as many variations on these designs already exist, and research continues to evolve. They do, however, demonstrate the possibility of performing quantum computation using spin-defined qubits, as well as the successes of modern condensed matter theory and experimentation in describing and harnessing a wide range of quantum effects. Each qubit presents unique advantages that can push forward the field of quantum computation, as well as challenges which invite innovations in basic quantum science to overcome.

\begin{thebibliography}{99}
I would like to acknowledge Kaelyn Ferris and Samuel Johnson, as well as Professor Jesse Berezovsky of Case Western Reserve University, for their helpful presentations introducing quantum computing and spin qubits.
\end{thebibliography}
\bibliography{spin-qubits.bib}{}

\end{document}
